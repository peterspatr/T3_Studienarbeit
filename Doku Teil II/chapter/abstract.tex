\chapter*{Kurzfassung} %*-Variante sorgt dafür, das Abstract nicht im Inhaltsverzeichnis auftaucht

%Problemstellung
%
%Ziel der Arbeit
%
%Vorgehen und angewandte Methoden
%
%Konkrete Ergebnisse der Arbeit, am besten mit quantitativen Angaben
 

Diese Studienarbeit wurde im Rahmen der sechsten Akademiephase angefertigt. Im ersten Teil der Arbeit wurden verschiedene Spieltheorie mit einander verglichen  .
Aufbauend auf den Erkenntnissen und Ergebnissen der ersten Studienarbeit lag der Schwerpunkt diesmal auf der praktischen Umsetzung eines Vier-Gewinnt-Roboters. Ziel war es, einen Roboter zu entwickeln, der in der Lage ist, das bekannte Spiel „Vier gewinnt“ eigenständig zu spielen.

Für die Realisierung des Projekts wurde ein LEGO Spike Prime Set verwendet. Die Konstruktion des Roboters besteht aus verschiedenen zusammengesetzten LEGO-Bauelementen. Vereinzelt wurden auch Elemente selbst gezeichnet und mit dem 3D-Drucker angefertigt.  
Die Steuerung und Überwachung der Aktoren, wie zum Beispiel der Motoren und Sensoren, erfolgt über den LEGO Spike Hub. Dieser zentrale Mikrocontroller verarbeitet die eingehenden Sensordaten und steuert die Bewegungen des Roboters entsprechend der programmierten Logik.

Durch die Kombination aus mechanischer Konstruktion und intelligenter Programmierung entstand so ein funktionsfähiger Prototyp, der die gestellten Anforderungen erfüllt und das Spiel eigenständig ausführen kann.

\clearpage

\chapter*{Abstract} %*-Variante sorgt dafür, das Abstract nicht im Inhaltsverzeichnis auftaucht

English translation of the \glqq Kurzfassung\grqq.

\clearpage