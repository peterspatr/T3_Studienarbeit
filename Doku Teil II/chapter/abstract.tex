\chapter*{Kurzfassung} %*-Variante sorgt dafür, das Abstract nicht im Inhaltsverzeichnis auftaucht

%Problemstellung
%
%Ziel der Arbeit
%
%Vorgehen und angewandte Methoden
%
%Konkrete Ergebnisse der Arbeit, am besten mit quantitativen Angaben
 

Diese Studienarbeit wurde im Rahmen der sechsten Akademiephase angefertigt. Im ersten Teil der Arbeit wurden verschiedene Spieltheorien miteinander verglichen.
Aufbauend auf den Erkenntnissen der ersten Studienarbeit, lag der Schwerpunkt diesmal in der praktischen Umsetzung eines Vier-Gewinnt-Roboters. Das Ziel dieser Arbeit bestand darin, einen Roboter zu entwickeln, der eigenständig Spielzüge beim Spiel „Vier gewinnt“ ausführen kann und somit in der Lage ist, gegen einen anderen Roboter anzutreten.

Für die Realisierung des Projekts wurde ein LEGO Spike Prime Set verwendet. Die Konstruktion des Roboters besteht aus verschiedenen zusammengesetzten LEGO-Bauelementen. Vereinzelt wurden auch Elemente selbst entworfen und mittels 3D-Drucker angefertigt.  
Die Steuerung und Überwachung der Aktoren, wie zum Beispiel der Motoren und Sensoren, erfolgt mithilfe eines LEGO Spike Hub. Dieser Mikrocontroller verarbeitet die eingehenden Sensordaten und steuert die Bewegungen des Roboters entsprechend der programmierten Logik.

Die Programmierung erfolgte in der Sprache MircoPython in der LEGO Spike App.
Das Kernstück des Programms ist der Alpha-Beta-Algorithmus, mit ihm wird der nächste optimale Zug berechnet.
Durch die Kombination aus mechanischer Konstruktion und programmierter Software entstand ein funktionsfähiger Prototyp, der die gestellten Anforderungen erfüllt und einen Spielzug eigenständig ausführen kann.

\clearpage

\chapter*{Abstract} %*-Variante sorgt dafür, das Abstract nicht im Inhaltsverzeichnis auftaucht

This student research project was completed as part of the sixth academy phase. In the first part of the thesis, different game theories were compared with each other.
Building on the knowledge gained from the first dissertation, this time the focus was on the practical realisation of a four-in-a-row robot. The aim of this work was to develop a robot that can independently perform moves in the game ‘Four Wins’ and is therefore able to compete against another robot.

A LEGO Spike Prime set was used to realise the project. The construction of the robot consists of various assembled LEGO building elements. Some of the elements were also designed in-house and produced using a 3D printer.  
The actuators, such as the motors and sensors, are controlled and monitored using a LEGO Spike Hub. This microcontroller processes the incoming sensor data and controls the robot's movements according to the programmed logic.

Programming was carried out in the MircoPython language in the LEGO Spike app.
The core of the programme is the alpha-beta algorithm, which is used to calculate the next optimal move.
The combination of mechanical construction and programmed software resulted in a functional prototype that fulfils the requirements and can execute a move independently.

Translated with DeepL.com (free version)

\clearpage