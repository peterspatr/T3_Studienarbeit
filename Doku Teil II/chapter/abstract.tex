\chapter*{Kurzfassung} %*-Variante sorgt dafür, das Abstract nicht im Inhaltsverzeichnis auftaucht

%Problemstellung
%
%Ziel der Arbeit
%
%Vorgehen und angewandte Methoden
%
%Konkrete Ergebnisse der Arbeit, am besten mit quantitativen Angaben
 

Diese Studienarbeit wurde im Rahmen der sechsten Akademiephase angefertigt. Im ersten Teil der Arbeit wurden verschiedene Spieltheorien miteinander verglichen.
Aufbauend auf den Erkenntnissen der ersten Studienarbeit, lag der Schwerpunkt diesmal in der praktischen Umsetzung eines Vier-Gewinnt-Roboters. Das Ziel dieser Arbeit bestand darin, einen Roboter zu entwickeln, der eigenständig Spielzüge beim Spiel „Vier gewinnt“ ausführen kann und somit in der Lage ist, gegen einen anderen Roboter anzutreten.

Für die Realisierung des Projekts wurde ein LEGO Spike Prime Set verwendet. Die Konstruktion des Roboters besteht aus verschiedenen zusammengesetzten LEGO-Bauelementen. Vereinzelt wurden auch Elemente selbst entworfen und mittels 3D-Drucker angefertigt.  
Die Steuerung und Überwachung der Aktoren, wie zum Beispiel der Motoren und Sensoren, erfolgt mithilfe eines LEGO Spike Hub. Dieser Mikrocontroller verarbeitet die eingehenden Sensordaten und steuert die Bewegungen des Roboters entsprechend der programmierten Logik.

Die Programmierung erfolgte in der Sprache MircoPython in der LEGO Spike App.
Das Kernstück des Programms ist der Alpha-Beta-Algorithmus, mit ihm wird der nächste optimale Zug berechnet.
Durch die Kombination aus mechanischer Konstruktion und programmierter Software entstand ein funktionsfähiger Prototyp, der die gestellten Anforderungen erfüllt und einen Spielzug eigenständig ausführen kann.

\clearpage

\chapter*{Abstract} %*-Variante sorgt dafür, das Abstract nicht im Inhaltsverzeichnis auftaucht

English translation of the \glqq Kurzfassung\grqq.

\clearpage