\chapter{Mikropython}


MicroPython ist eine für Mikrocontroller optimierte Version der Programmiersprache Python. Im Gegensatz zur klassischen Desktop-Variante ist MicroPython ressourcenschonend und kann direkt auf eingebetteten Systemen wie dem Großen LEGO® Technic Hub ausgeführt werden. LEGO® Education SPIKE™ Prime nutzt MicroPython als Grundlage für die textbasierte Programmierung außerhalb der Standard-GUI, was besonders fortgeschrittenen Nutzern und Entwicklern zusätzliche Freiheiten eröffnet.

Der Große LEGO® Technic Hub ist mit einem leistungsfähigen 32-Bit ARM Cortex-M4 Mikrocontroller ausgestattet, der mit einer Taktfrequenz von 100 MHz arbeitet. Darüber hinaus verfügt der Hub über 1 MB Flash-Speicher und 128 KB RAM, was für viele schulische und experimentelle Anwendungen mehr als ausreichend ist. Sechs Anschlüsse für Motoren und Sensoren sowie Bluetooth Low Energy (BLE) zur drahtlosen Kommunikation machen den Hub vielseitig einsetzbar.

Durch den Einsatz von MicroPython lassen sich komplexere Algorithmen – wie etwa Spiel-KI (z.\,B. Minimax bei 4-Gewinnt), Sensorfusion oder datenbasierte Steuerungen – effizient umsetzen. Der Zugriff auf Motoren, Sensoren und interne Funktionen ist über eine dedizierte MicroPython-API möglich, was die Programmierung stark flexibilisiert. Insgesamt verbindet MicroPython die Einfachheit von Python mit der unmittelbaren Steuerung von Hardwarekomponenten und macht den Großen Technic Hub zu einer leistungsfähigen Plattform für robotische und algorithmische Experimente.
