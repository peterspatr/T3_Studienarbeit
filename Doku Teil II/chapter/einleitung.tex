\chapter{Einleitung}

Das allseits bekannte klassische Gesellschaftsspiel \glqq Vier Gewinnt\grqq~erfreut sich seit sehr vielen Jahren großer Beliebtheit bei Alt und Jung. Es ist ein strategisches Zweipersonenspiel mit sehr einfachen Regeln, aber mit einer erstaunlichen Spieltiefe. Im Zeitalter der Industrie 4.0 und Digitalisierung sollen immer mehr Abläufe automatisiert werden. Dadurch stellt sich die Frage, wie sich das Spiel „Vier Gewinnt“ mithilfe von Robotik und Algorithmus automatisieren lässt.    

Im ersten Teil der Studienarbeit T3\_3100 wurden zunächst die spieltheoretischen Grundlagen untersucht. Hierbei wurden zentrale Konzepte wie dominante Strategien, Nash-Gleichgewichte und das Minimax-Prinzip näher analysiert. Ein besonderes Augenmerk wurde auf die Entwicklung von Algorithmen gelegt. Der Fokus lag hierbei auf dem Alpha-Beta-Algorithmus.

Aufbauend auf der ersten Arbeit geht es in diesem Teil der Studienarbeit um den Entwurf und um die Realisierung eines Roboters, der eigenständig Spielzüge im Spiel \glqq Vier Gewinnt\grqq~ ausführen kann. Ziel dieser Arbeit ist es, einen Roboter mithilfe von LEGO Spike Prime zu konstruieren und zu programmieren, der in der Lage ist, selbstständig das Spielfeld abzufahren und gegnerische Spielsteine zu erkennen. Auf diesen Fähigkeiten aufbauend soll der Roboter in der Lage sein, mithilfe eines Algorithmus den optimalen Spielzug zu berechnen und diesen eigenständig auszuführen.

Diese Arbeit beinhaltet sowohl mechanische und elektrische Entwicklung als auch das Entwerfen der Software. Ein besonderes wichtig hierbei ist das Zusammenspiel zwischen Hardware und Software, die es dem Roboter ermöglichen, flexibel und zuverlässig auf verschiedene Spielsituationen reagieren zu können.

Die Studienarbeit zielt auch darauf ab, eine Verbindung zwischen theoretischen Konzepten und der praktischen Anwendung in der Robotik zu demonstrieren. Sie ist dabei in folgende Kapitel unterteilt.

\begin{itemize}
	\item \textbf{Grundlagen:} Im Kapitel \glqq Grundlagen\grqq~ werden die zentralen Begriffe und Methoden vorgestellt.
	\item \textbf{Vorgehen:} Im zweiten Kapitel \glqq Vorgehen\grqq~ wird auf die Anforderung an das System eingegangen. Darüber hinaus werden in diesem Abschnitt auch das Konzept und die Planung erläutert.
	\item \textbf{Umsetzung:} Im letzten Kapitel \glqq Umsetzung\grqq~ geht es um die praktische Realisierung des Roboters. Hierbei wird auf den mechanischen Aufbau, sowie die Funktion der Software eingegangen. 
\end{itemize}