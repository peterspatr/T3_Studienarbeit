\chapter{ Umsetzung}
\label{cha:Umsetzung}
In diesem Kapitel steht die praktische Umsetzung des Roboters im Fokus. Dabei wird auf den Aufbau eingegangen, ebenso wie auf die Funktionsweise der Software. 
%Je nach Art der Arbeit kann diese Kapitelüberschrift auch \glqq Ergebnisse\grqq~lauten, z.~B. bei rein messtechnischen Aufgaben.

%Beschreibung der Umsetzung des zuvor gewählten Vorgehens (theoretische Untersuchung, Erhebungen, Durchführung von Experimenten, Prototypenaufbau, Implementierung eines Prozesses, etc.).

%Verifikation anhand der zuvor erarbeiteten Anforderungen und Validierung in Bezug auf das zuvor gestellte Ziel. Diskussion der Ergebnisse. Spätestens hier auch auf die Zuverlässigkeit der gewonnenen Erkenntnisse eingehen (z.~B. anhand der Genauigkeit von Messergebnissen).
\section{Mechanischer Aufbau}
Für die Umsetzung des 4-Gewinnt-Roboters wurde eine mechanische Konstruktion gewählt, die es erlaubt, das Spielfeld zu scannen sowie Chips gezielt in eine Spalte einzuwerfen. Der Aufbau umfasst drei Winkelmotoren, einen Farbsensor und einen Drucktaster. Im Folgenden werden die einzelnen Komponenten detailliert beschrieben. Dabei beziehen sich die Ziffern der Aufzählung der einzelnen Sensoren und Aktoren auf die Abbildung \ref{fig:seitenansicht-links} und Abbildung \ref{fig:seitenansicht-rechts}.

Die mechanische Konstruktion basiert auf die Idee einem kartesischen Koordinatensystem, bei dem der Farbsensor durch die Kombination aus horizontaler und vertikaler Bewegung jede Spielfeldposition präzise anfahren kann. Das System erlaubt eine vollautomatische Spielweise.

\subsection{Einbindung von Aktorik}
\begin{enumerate}
	\item \textbf{Horizontalantrieb}\\
	Der horizontale Antrieb des Farbsensors erfolgt über einen Winkelmotor. Dieser ist dafür zuständig, die Spielfeldspalten nacheinander anzufahren. Der Motor ist mit einer Achse verbunden, welche zwei Räder antreiben. Die Bewegung erfolgt in gleichmäßigen Schritten: Eine Drehung um exakt 72 Grad bewegt den Roboter um eine Spalte weiter. Diese Schrittweite wurde so gewählt, dass sie der Breite einer Spalte im Spielfeld entspricht. Dadurch ist eine exakte Positionierung des Sensors über jeder Spalte möglich, ohne dass zusätzliche Sensoren zur Positionsbestimmung notwendig sind. 
	\item \textbf{Vertikalantrieb}\\
	 Um das Spielfeld auch in vertikaler Richtung abfahren zu können, ist der Farbsensor an einer Kette montiert. Diese Kette wird durch einen Winkelmotor angetrieben. Der Sensor ist an einem mittleren Segment der Kette befestigt und fährt beim Drehen der Kette entsprechend auf und ab. Ein Schritt des Motors um 95 Grad bewegt den Sensor um genau eine Spielfeldhöhe weiter. Auf diese Weise können sämtliche sechs Reihen der aktuellen Spalte nacheinander abgescannt werden. Die Rückwärtsbewegung der Kette erlaubt es, den Sensor wieder nach unten zu fahren.
	\item \textbf{Chipauswerfer}\\
	Das Einwerfen des eigenen Spielsteins erfolgt ebenfalls über einen Winkelmotor. An diesem Motor ist eine Stange montiert, die bei einer vollständigen Umdrehung einen Spielchip aus dem Vorratsmagazin (mit der Software Fusion360 konstrukiert und 3D-gedruckt) in die gewünschte Spalte stößt. Nach der Auslösung kann ein neuer Chip in die Abschussposition nachrutschen. In der Software ist eine Wartezeit nach dem Auslösen eingebaut, damit der Chip sicher im Spielfeld ankommt, bevor die nächste Aktion beginnt.


\subsection{Einbindung von Sensorik}

	\item \textbf{Startsignal}\\
	Um dem Roboter mitzuteilen, dass er den nächsten Zug starten kann wurde ein Kraftsensor angebracht. Dieser befindet sich an der Vorderseite des Roboters. Sobald der Spieler den Sensor leicht berührt, wird ein Signal ausgelöst und der Prozess startet. 
	\item \textbf{Spielfeldscan }\\
	 Für die Farberkennung des Spielfeldes wurde ein LEGO-Farbsensor verwendet, der über die oben beschriebene Kettenkonstruktion vertikal verfahrbar ist. Die Farbmessung erfolgt jeweils in der Mitte eines Spielfeldes. Der Sensor erkennt RGB-Werte (in diesem Projekt benutzt: Rot, Gelb oder Leer). Der Abstand zwischen Sensor und Spielfeld beträgt etwa 7 mm. Dieser Wert hat sich als optimal für zuverlässige Farbmessung erwiesen.
\end{enumerate}

\begin{figure}[H]
	\centering
	\begin{minipage}[b]{0.39\linewidth}
		\centering
		\includegraphics[width=\linewidth]{images/188CF006-D571-4F2D-9337-8C4BDD7DAAEF_1_105_c}
		\caption{Seitenansicht links}
		\label{fig:seitenansicht-links}
	\end{minipage}
	\hfill
	\begin{minipage}[b]{0.39\linewidth}
		\centering
		\includegraphics[width=\linewidth]{images/DAE26A50-277E-4C6B-96A3-F2DE2CC9C004_1_105_c}
		\caption{Seitenansicht rechts}
		\label{fig:seitenansicht-rechts}
	\end{minipage}
	\fbox{%
	\begin{minipage}{0.5\linewidth}
		\small
		\textbf{Legende:}\\ \textbf{1} = Motor für horizontale Bewegung, \\ \textbf{2} = Motor für vertikale Bewegung, \\ \textbf{3} = Motor für Steineinwurf,\\ \textbf{4} = Kraft-/Touchsensor
		\textbf{5} = Farbsensor,
	\end{minipage}}%
\end{figure}


\section{Software}
Das Kapitel Software beschreibt ausführlicher, wie der Vier-Gewinnt-Roboter programmiert wird. Die Software ist das Herzstück des Roboters und spielt eine entscheidende Rolle für seine autonome Teilnahme am Spiel.

\subsection{Ablauf der Software}
Das in der Abbildung \ref{fig:Flussdiagramm} dargestellte Flussdiagramm zeigt grafisch den groben Ablauf der Software. \newline
Zu Beginn befindet sich das System im Wartezustand und wartet auf die Betätigung des Touch-/Kraftsensors. Dieser signalisiert den Start eines neuen Spielzugs. Sobald der Sensor aktiviert wurde, fährt der Roboter zur ersten Spalte des Spielfelds.
Anschließend scannt das System die einzelnen Spalten des Spielfelds ab, um den neuen gegnerischen Stein zu entdecken. Dabei wird das interne Spielfeld aktualisiert, bis ein neuer gegnerischer Stein erkannt wurde oder alle 7 Spalten gescannt wurden.

Im nächsten Schritt prüft die Software das interne Spielfeld und berechnet mithilfe des Alpha-Beta-Algorithmus den optimalen Spielzug.
Dabei ist es nicht von Bedeutung, ob ein gegnerischer Stein erkannt wurde oder nicht. Da zu Beginn des Spiels noch gar keine Steine im Feld vorhanden sind.
Der Roboter platziert nach der Berechnung den eigenen Spielstein an der berechneten optimalen Position.

Im Anschluss prüft die Software, ob das Spiel gewonnen wurde. Ist das der Fall, so wird eine Glückwunschmeldung und ein Ton ausgegeben. Falls kein Gewinn vorliegt, wird einfach das aktuelle Spielfeld auf der Konsole in der LEGO Spike App ausgegeben und der Prozess beginnt von vorne in der Warteposition mit dem Warten auf die nächste Sensoraktivierung. 




\tikzstyle{startstop} = [ellipse, draw, fill=gray!10, minimum width=3cm, minimum height=1cm]
\tikzstyle{io} = [trapezium, trapezium left angle=70, trapezium right angle=110, draw, fill=blue!10, minimum width=3cm, minimum height=1cm]
\tikzstyle{process} = [rectangle, draw, fill=orange!10, minimum width=3cm, minimum height=1cm]
\tikzstyle{decision} = [diamond, draw, fill=green!10, aspect=2, minimum width=3cm, minimum height=1cm]
\tikzstyle{arrow} = [thick,->,>=Stealth]

\begin{figure}[H]

\begin{tikzpicture}[node distance=2cm]
	\node (start) [startstop] {Start};
	\node (wait) [process, below of=start] {Warten auf Sensor-Aktivierung};
	\node (move) [process, below of=wait] {Spielfeld anfahren};
	\node (scan) [process, below of=move] {Spalten scannen};
	\node (update) [process, below of=scan] {Board-Update};
	\node (opponent) [decision, below of=update, yshift=-0.7cm] {Neuer gegnerischer Stein?};
	\node (calc) [process, below of=opponent, yshift=-0.7cm] {Optimalen Zug berechnen (Alpha-Beta)};
	\node (motor) [process, below of=calc] {Motorsteuerung: Stein platzieren};
	\node (win) [decision, below of=motor, yshift=-0.7cm] {Gewonnen?};
	\node (congrats) [startstop, right of=win, xshift=5cm] {Glückwunsch! Spiel gewonnen};
	\node (print) [process, below of=win] {Spielfeld ausgeben};

	
	\draw [arrow] (start) -- (wait);
	\draw [arrow] (wait) -- (move);
	\draw [arrow] (move) -- (scan);
	\draw [arrow] (scan) -- (update);
	\draw [arrow] (update) -- (opponent);
	\draw [arrow] (opponent) -- node[right] {Ja/Nein} (calc);
	\draw [arrow] (calc) -- (motor);
	\draw [arrow] (motor) -- (win);
	\draw [arrow] (win) -- node[above] {Ja} (congrats);
	\draw [arrow] (win) -- node[right] {Nein} (print);
	\draw [arrow] (congrats) |- (print);
	\draw [arrow] (print.west) -| ($(wait.west)+(-1.5,0)$) -- (wait.west);
	
\end{tikzpicture}

	\caption[Flussdiagramm]{Flussdiagramm der Software}
\label{fig:Flussdiagramm}
\end{figure}
\section{Programmlogik}

In diesem Kapitel wird die Softwarestruktur des 4-Gewinnt-Roboters systematisch beschrieben. Da der vollständige Quellcode eine Vielzahl an Funktionen, Hilfsroutinen und technischen Details umfasst, wird in diesem Kapitel aus Gründen der Übersichtlichkeit nicht jede einzelne Codezeile dargestellt und erläutert.
Stattdessen liegt der Fokus auf den wesentlichen Programmabschnitten, die das Spielverhalten maßgeblich bestimmen. Der vollständige Code kann aus dem Anhang entnommen werden.

Zur besseren Nachvollziehbarkeit der Funktionsweise wird die Darstellung in zwei Teile gegliedert:

\begin{itemize}
	\item Zunächst wird ausschließlich der \textbf{Algorithmus} zur Spielentscheidung detailliert erläutert. Dabei handelt es sich um den Minimax-Algorithmus mit Alpha-Beta-Pruning.
	
	\item Im Anschluss wird das \textbf{Hauptprogramm} vorgestellt, das alle Bestandteile miteinander verknüpft. Es steuert den Ablauf über das Spielfeld-Scanning bis zum Berechnen und Ausführen des eigenen Spielzugs. Dabei werden sowohl Sensoren als auch Motoren angesprochen und der Entscheidungsalgorithmus eingebunden.
\end{itemize}

Diese Zweiteilung erlaubt es, sowohl die algorithmische Tiefe als auch die technische Umsetzung getrennt zu betrachten und anschließend im Gesamtkontext zu verstehen.

Desweiteren wird im Unterkapitel 4.7 die konkrete Umsetzung der Spielsteuerung und Entscheidungslogik für den 4-Gewinnt-Roboter beschrieben. Der Fokus liegt auf der softwareseitigen Realisierung unter Berücksichtigung der limitierten Hardware-Ressourcen des LEGO Spike Prime Hubs. Es werden hierbei zentrale Entwurfsentscheidungen erläutert.


\subsection{Algorithmus}

Ein zentraler Bestandteil des 4-Gewinnt-Roboters ist die Entscheidungsfindung durch einen algorithmischen Spielbaum. Dieser wird mit dem bekannten Minimax-Algorithmus unter Verwendung von Alpha-Beta-Pruning realisiert. Ziel ist es, basierend auf dem aktuellen Spielfeldzustand den optimalen Zug für den Roboter zu berechnen.\\
Der Algorithmus bewertet mögliche Züge bis zu einer bestimmten Tiefe im Spielbaum und trifft Entscheidungen, die langfristig zum Sieg führen können oder gegnerische Gewinnzüge verhindern.

\textbf{Spielfeld und Spielerdefinition:}\\
Im Vorfeld des Algorithmus ist festgelegt, welches Symbol der Algorithmus (Roboter) spielt:

\begin{lstlisting}[style=pythonstyle]
	my_piece = 1  # 1 = YELLOW (AI player), -1 = RED
	opponent_piece = -my_piece
\end{lstlisting}

Dabei entspricht 1 dem gelben Spielstein (Roboter), -1 dem roten Spielstein (Gegner). Diese numerische Darstellung vereinfacht die Bewertung und das Vergleichen der Felder im Spielfeld.

\textbf{Bewertungsfunktion:}\\
Der Algorithmus benötigt eine Bewertungsfunktion, die die Qualität eines Spielzustands abschätzt. Dies geschieht durch eine Heuristik, die mögliche Gewinnlinien zählt und bewertet.
Die Bewertungsfunktion basiert auf der Idee, sogenannte „Fenster“ (Ausschnitte aus 4 Feldern) im Spiel zu analysieren und zu beurteilen, wie viele Steine der Roboter bzw. des Gegners in diesen Fenstern enthalten sind: \newpage

\begin{lstlisting}[style=pythonstyle]
	def evaluate_window(window, player):
			opp_player = opponent_piece 
		if player == my_piece else 	my_piece
			score = 0
		if window.count(player) == 4:
			score += 100
		elif window.count(player) == 3 and window.count(0) == 1:
			score += 5
		elif window.count(player) == 2 and window.count(0) == 2:
			score += 2
		if window.count(opp_player) == 3 and window.count(0)==1:
			score -= 4
		return score
\end{lstlisting}

Diese Funktion bewertet sowohl offensive als auch defensive Situationen. Ein Fenster mit drei eigenen Steinen und einem leeren Feld wird positiv bewertet, ein Fenster mit drei gegnerischen Steinen und einem leeren Feld hingegen negativ, um Bedrohungen abzuwehren.

Die Hauptfunktion zur Bewertung des gesamten Spielfeldes aggregiert alle horizontalen, vertikalen und diagonalen Fenster:

\begin{lstlisting}[style=pythonstyle]
	def evaluate(board):
		score = 0
		center_array = [board[r][field_width//2]
		 for r in range(field_height)]
			center_count = center_array.count(my_piece)
			score += center_count * 3
\end{lstlisting}

Zunächst werden die mittleren Spalten stärker gewichtet, da sie strategisch wichtiger sind.

Anschließend werden alle Zeilen, Spalten und Diagonalen analysiert:

\begin{lstlisting}[style=pythonstyle]
	for r in range(field_height):
		row_array = [board[r][c] for c in range(field_width)]
	for c in range(field_width - 3):
		window = row_array[c:c+4]
		score += evaluate_window(window, my_piece)
		score -= evaluate_window(window, opponent_piece)
\end{lstlisting}

Diese Schleifen bilden das heuristische Fundament für die spätere Entscheidungsfindung.

\textbf{Minimax mit Alpha-Beta-Pruning:}\\
Die Hauptentscheidung trifft der Minimax-Algorithmus. Dabei wird rekursiv der Spielbaum aufgebaut, wobei sich der Algorithmus abwechselnd in die Rolle des Roboters („maximizing player“) und des Gegners („minimizing player“) versetzt. Um die Effizienz zu steigern, wird Alpha-Beta-Pruning genutzt. Dabei werden Äste im Spielbaum verworfen, wenn sie nachweislich zu schlechteren Ergebnissen führen.

Der Einstiegspunkt ist:

\begin{lstlisting}[style=pythonstyle]
	def alpha_beta(board, depth, alpha, beta, maximizing_player)
\end{lstlisting}

Zuerst wird geprüft, ob der aktuelle Zustand bereits im Transposition Tabelle gespeichert ist. Das ist ein Cache zur Vermeidung redundanter Berechnungen:

\begin{lstlisting}[style=pythonstyle]
	key = (board_hash(board, maximizing_player), depth)
	if key in transposition_table:
	return transposition_table[key]
\end{lstlisting}

Anschließend erfolgt eine Prüfung: Ist der Zug eine Gewinnsituation, oder wurde die maximale Tiefe erreicht?

\begin{lstlisting}[style=pythonstyle]
	valid_locations = [col for col in range(field_width)
	 if is_valid_location(board, col)]
		terminal = winning_move(board, my_piece) or winning_move	(board, opponent_piece) or len(valid_locations) == 0
	if depth == 0 or terminal:
	...
\end{lstlisting}

Falls ja, gibt die Funktion eine Bewertung zurück. Andernfalls wird der Spielbaum weiter durchlaufen.

\textbf{Maximierender Spieler (Roboter):}\\
Hier versucht der Algorithmus, die maximal erreichbare Bewertung zu finden und prüft regelmäßig, ob das aktuelle Ergebnis besser ist als die bisherige beste Option. Wenn \texttt{alpha >= beta}, wird der restliche Baum abgeschnitten (Pruning).

\begin{lstlisting}[style=pythonstyle]
	if maximizing_player:
		value = -float('inf')
	for col in valid_locations:
		...
	new_score = alpha_beta(..., False)[1]
	if new_score > value:
		value = new_score
		best_col = col
		alpha = max(alpha, value)
	if alpha >= beta:
	break
\end{lstlisting}

\textbf{Minimierender Spieler (Gegner):}\\
Analog erfolgt das Vorgehen für den Gegner:

\begin{lstlisting}[style=pythonstyle]
	else:
	value = float('inf')
	for col in valid_locations:
		...
	new_score = alpha_beta(..., True)[1]
	if new_score < value:
		value = new_score
		best_col = col
		beta = min(beta, value)
	if beta <= alpha:
	break
\end{lstlisting}

Am Ende wird das Ergebnis in der Transpositionstabelle gespeichert und zurückgegeben:

\begin{lstlisting}[style=pythonstyle]
	transposition_table[key] = result
	return result
\end{lstlisting}

\textbf{Dynamische Suchtiefe:}\\
Je nach Spielphase kann es sinnvoll sein, tiefer oder flacher zu suchen. Zu Beginn reicht eine niedrige Tiefe, da viele Züge möglich sind. In späteren Phasen erhöht sich die Tiefe:

\begin{lstlisting}[style=pythonstyle]
	def get_dynamic_depth(board):
		empty = sum(row.count(0) for row in board)
		if empty > 30:
			return 3
			else:
			return 4
\end{lstlisting}

Diese dynamische Anpassung balanciert Spielstärke und Rechenzeit optimal.

%Der eingesetzte Minimax-Algorithmus mit Alpha-Beta-Pruning stellt das strategische Herzstück des 4-Gewinnt-Roboters dar. Durch gezielte Bewertung von Spielpositionen, Berücksichtigung gegnerischer Drohungen und dynamische Tiefenanpassung kann der Roboter selbstständig Züge planen, Gefahren abwehren und letztlich siegreich agieren. Die Verwendung eines Transpositionstables beschleunigt dabei die Entscheidungsfindung, indem bereits analysierte Spielsituationen nicht erneut bewertet werden müssen. Das Ergebnis ist ein hochgradig effektives Entscheidungsverfahren für ein strategisches Spiel wie 4-Gewinnt.


\section{Hauptprogramm}

Nachdem der Algorithmus erläutert wurde und in Kapitel 4.1 die Konstruktion und Ansteuerung des Roboters aufgezeigt wurde, beschreibt dieses Kapitel den Gesamtablauf des Programms. Dabei steht im Fokus, wie die Spielfelderkennung, Algorithmus und Ausführungsschritte zu einem vollständigen Spielzug kombiniert werden.

\textbf{Initialisierung:}\\
Zu Beginn wird das Spielfeld als leere Matrix angelegt. Zusätzlich wird eine Kopie gespeichert, um Änderungen im Vergleich zur vorherigen Runde erkennen zu können.

\begin{lstlisting}[style=pythonstyle]
	board = [[0 for _ in range(field_width)] 
	for _ in range(field_height)]
	last_board = [row[:] for row in board]
\end{lstlisting}

\textbf{Warten auf Eingabe durch den Spieler:}\\
Bevor der Roboter mit dem Scannen des Spielfeldes beginnt, wartet er auf eine Aktivierung des Drucksensors am Port C.

\begin{lstlisting}[style=pythonstyle]
	print("Waiting for sensor at port C...")
	while not sensor_activated():
	time.sleep(0.1)
\end{lstlisting}

\textbf{Positionierung an der Startspalte:}\\
Der Roboter fährt an die rechte Spielfeldseite (Spalte 0), um von dort den Scan zu beginnen.

\begin{lstlisting}[style=pythonstyle]
	motor.run_for_degrees(port.D, 198, 170)
	time.sleep(1.5)
\end{lstlisting}

\textbf{Scannen des Spielfelds:}\\
Von rechts nach links wird jede Spalte analysiert. Dabei wird der Farbsensor in die erste freie Zeile der Spalte bewegt:

\begin{lstlisting}[style=pythonstyle]
	motor.run_for_degrees(port.E, move_distance_e * (free_row), speed_E)
	detected_color = color_sensor.color(port.B)
	update_board(free_row, matrix_col, detected_color)
	motor.run_for_degrees(port.E, -move_distance_e * free_row, speed_E)
\end{lstlisting}

Wird ein neuer gegnerischer Spielstein erkannt, wird seine Position gespeichert und der Scan abgebrochen:

\begin{lstlisting}[style=pythonstyle]
	if (last_board[free_row][matrix_col] == 0 and
	board[free_row][matrix_col] == opponent_piece):
		opponent_piece_found = True
		opponent_col = col
\end{lstlisting}

\textbf{Berechnung des Spielzugs:}\\
Die Tiefe der Suche wird dynamisch abhängig vom Spielstand gewählt. Anschließend wird der beste Spielzug mit Minimax und Alpha-Beta-Pruning berechnet:

\begin{lstlisting}[style=pythonstyle]
	dynamic_depth = get_dynamic_depth(board_numeric)
	best_col, _ = alpha_beta(
	board_numeric,
	depth=dynamic_depth,
	alpha=-float('inf'),
	beta=float('inf'),
	maximizing_player=(my_piece == 1)
	)
\end{lstlisting}

\textbf{Ausführen des Spielzugs:}\\
Zuerst wird die physische Zielspalte berechnet und der Roboter dorthin bewegt:

\begin{lstlisting}[style=pythonstyle]
	physical_target_col = field_width - 1 - best_col
	motor.run_for_degrees(port.D, -move_distance_d * physical_target_col, speed_D)
\end{lstlisting}

Danach wird ein Spielstein mithilfe des Motors A ausgeworfen:

\begin{lstlisting}[style=pythonstyle]
	motor.run_for_degrees(port.A, -360, speed_A)
	time.sleep(3)
\end{lstlisting}

Das Spielfeld wird nach dem Wurf aktualisiert:

\begin{lstlisting}[style=pythonstyle]
	board[best_row][best_col] = my_piece
	last_board = [row[:] for row in board]
\end{lstlisting}

Anschließend erfolgt eine Prüfung auf einen möglichen Spielsieg:

\begin{lstlisting}[style=pythonstyle]
	if winning_move(board, my_piece):
		print(" Congratulations! The robot has WON the game!")
		print_board(board)
		sound.beep(440, 1000000, 100)
		break
\end{lstlisting}

\textbf{Zurückfahren in Ausgangsposition:}\\

Unabhängig vom Spielausgang kehrt der Roboter an seine Startposition zurück:

\begin{lstlisting}[style=pythonstyle]
	motor.run_for_degrees(port.D, -199, speed_D)
\end{lstlisting}

\textbf{Warten auf die nächste Runde:}\\
Abschließend wird auf das Loslassen des Drucksensors gewartet, bevor ein neuer Zyklus beginnt:

\begin{lstlisting}[style=pythonstyle]
	while sensor_activated():
	time.sleep(0.1)
\end{lstlisting}


%Die Ablaufsteuerung des Hauptprogramms ist zyklisch aufgebaut und gewährleistet einen strukturierten Spielverlauf: Der Roboter wartet auf das Startsignal, scannt das Spielfeld, berechnet den optimalen Spielzug und führt diesen präzise aus. Die Kombination aus Sensorik, Algorithmik und Motorsteuerung wird dabei durch eine klare Programmstruktur miteinander verbunden. Diese Trennung der Verantwortlichkeiten sorgt für Übersichtlichkeit, Erweiterbarkeit und Fehlerrobustheit.

\section{Ressourcenschonende Implementierung der Spielsteuerung und Entscheidungslogik}

Die Implementierung des Spielablaufs und der Entscheidungslogik wurde unter besonderer Berücksichtigung der eingeschränkten Ressourcen des LEGO Spike Prime Hub entwickelt. Durch gezielte Reduktion von unnötigen Berechnungen, Verwendung eines Gedächtnisses für das Spielfeld, dynamische Anpassung der Suchtiefe und einfache Ablaufsteuerung konnte ein vollständiger Spielzyklus umgesetzt werden, der sowohl strategisch leistungsfähig als auch technisch robust ist. Im Folgenden werden die zentralen Entwurfsentscheidungen aufgezeigt.

\subsection{Begrenzte Hardware-Ressourcen}
Der LEGO Spike Hub besitzt mit seinem 100MHz ARM Cortex-M4 Prozessor, 320 KB RAM und 1 MB Flash-Speicher eine stark begrenzte Hardwareausstattung \cite{lego2020techniclargehub}. Diese Ressourcen reichen für einfache Steuerungsaufgaben, setzen aber dem Einsatz komplexer Algorithmen wie Minimax enge Grenzen.
Diese Rahmenbedingungen erfordern eine möglichst effiziente und ressourcenschonende Programmstruktur. Daher wurde bewusst auf eine komplexe Multithread-Struktur verzichtet und stattdessen ein sequenzieller, wartender Ablauf gewählt.

\subsection{Zeitbasierte Steuerung mit \texttt{time.sleep()}}

Zur Koordination zwischen Sensorik, Motorik und internen Berechnungen wurde \texttt{time.sleep()} gezielt eingesetzt. Es erfüllt mehrere Aufgaben:

\begin{itemize}
	\item \textbf{Sicherstellung der mechanischen Stabilität:} Nach jeder Bewegung oder Farberkennung sorgt eine kurze Pause dafür, dass der Sensor sich mechanisch beruhigen kann und stabile Werte liefert.
	\item \textbf{Hardware-Synchronisierung:}  Vorgänge, wie etwa das Einwerfen eines Chips, benötigen eine kurze Wartezeit, die hardwareseitig nicht automatisch rückgemeldet wird. Durch gezielte Pausen wird so ein zuverlässiger Ablauf ohne ungewollte Überschneidungen erreicht.
	\item \textbf{Einfachheit:} In Abwesenheit von Interrupts auf dem Hub ist \texttt{time.sleep()} eine praktikable Lösung zur Ablaufsteuerung.
\end{itemize}

\subsection{Spielfeldvergleich zur Erkennung neuer Spielzüge}

Ein wesentlicher Optimierungsschritt liegt in der Verwendung eines ``Gedächtnisses'' über das vorherige Spielfeld. Zu Beginn jeder Spielrunde wird die aktuelle Matrix \texttt{board} mit dem gespeicherten Zustand \texttt{last\_board} verglichen. Ziel ist es, festzustellen, wo genau ein neuer gegnerischer Spielstein hinzugekommen ist, ohne jedes einzelne Feld vollständig neu scannen zu müssen:

\begin{lstlisting}[style=pythonstyle]
	if last_board[zeile][spalte] == 0 and board[zeile][spalte] == opponent_piece:
\end{lstlisting}

Dieser Vergleich erlaubt es, gezielt den neuen Zug des Gegners zu erkennen und den Scanvorgang direkt danach abzubrechen. Dies reduziert die benötigte Zeit pro Runde drastisch. Insbesondere im späteren Spielverlauf, wenn viele Felder bereits belegt sind.

\subsection{Spaltenweises Scannen statt Vollscan}

Anstatt das gesamte Spielfeld (6 Zeilen × 7 Spalten) vollständig zu scannen, wird nur von rechts nach links spaltenweise geprüft. Sobald ein neuer Spielstein entdeckt wurde, wird der Scan abgebrochen:

\begin{lstlisting}[style=pythonstyle]
	if opponent_piece_found:
		break
\end{lstlisting}

Diese Strategie basiert auf der Annahme, dass pro Runde exakt ein neuer gegnerischer Stein erscheint. Dadurch kann der Großteil des Spielfelds übersprungen werden, sobald der neue gegnerische Zug erkannt wurde. Dies reduziert die Anzahl der Motorbewegungen und gewinnt somit an Zeit.

\subsection{Dynamische Suchtiefe im Algorithmus}

Der Minimax-Algorithmus mit Alpha-Beta-Pruning wird verwendet, um den optimalen eigenen Spielzug zu berechnen. Um die Rechenlast dabei zu steuern, wird die maximale Suchtiefe dynamisch an die Spielsituation angepasst:

\begin{lstlisting}[style=pythonstyle]
	def get_dynamic_depth(board):
		empty = sum(row.count(0) for row in board)
		return 3 if empty > 30 else 4
\end{lstlisting}

In der Anfangsphase sind noch viele Züge möglich, was den Suchbaum exponentiell wachsen lässt. Eine flachere Suchtiefe (z.~B. 3) ist hier sinnvoll, da es ohnehin viele gleichwertige Optionen gibt. Im Endspiel hingegen sind nur noch wenige Felder frei, wodurch eine tiefere Suche (z.~B. 4) möglich und auch sinnvoll wird. Diese dynamische Anpassung balanciert Rechenzeit und Spielqualität optimal.

\subsection{Speicherung bewerteter Zustände (Transposition Table)}

Zur weiteren Reduktion der Rechenlast wird eine sogenannte Transposition Table eingesetzt. Diese speichert bereits bewertete Spielzustände in einer Hash-Tabelle, sodass doppelt auftretende Konstellationen nicht erneut berechnet werden müssen:

\begin{lstlisting}[style=pythonstyle]
	key = (board_hash(board, maximizing_player), depth)
	if key in transposition_table:
		return transposition_table[key]
\end{lstlisting}

Diese Technik ist besonders im mittleren Spielverlauf effektiv, da viele unterschiedliche Zugfolgen zu identischen Spielzuständen führen können.

\subsection{Minimale Boarddarstellung mit Ganzzahlen}

Das Spielfeld wird intern als Liste von Ganzzahlen (\texttt{-1}, \texttt{0}, \texttt{1}) dargestellt. Diese Codierung ist speicherarm, ermöglicht arithmetische Operationen (z.~B. Summieren zur Bewertung) und reduziert die Komplexität beim Kopieren und Vergleichen des Boards.

\section{Test und Versuchsauswertung}

Zur Überprüfung der Funktionalität und Spielstärke des entwickelten 4-Gewinnt-Roboters wurde eine umfangreiche Testreihe mit menschlichen Mitspielern durchgeführt. Ziel dieser Versuche war es, das Verhalten des Roboters in realen Spielsituationen zu beobachten, die Zuverlässigkeit der Spielfelderkennung zu bewerten und die Qualität des Entscheidungsalgorithmus praktisch zu prüfen.

\subsection{Durchführung und Ergebnisse der Testreihe}

Der Roboter wurde in der finalen Version gegen vier unterschiedliche Spieler getestet. 
Es wurden insgesamt 20 vollständige Partien gespielt. Die menschlichen Spieler handelten eigenständig und spielten mit realem Gewinninteresse.

Die Resultate der Testreihe lauten wie folgt:

\begin{itemize}
	\item \textbf{14 Siege des Roboters}
	\item \textbf{4 Unentschieden}
	\item \textbf{2 Niederlagen gegen menschliche Spieler}
\end{itemize}

Damit konnte der Roboter in 70\% der Partien gewinnen und blieb in 90\% der Fälle ungeschlagen.
\newpage

\subsection{Analyse der Unentschieden und Niederlagen}

Insgesamt sechs Partien wurden nicht gewonnen. Diese lassen sich in vier Unentschieden und zwei Niederlagen unterteilen. Beide Fälle sind technisch nachvollziehbar und lassen sich mit den Eigenschaften des Algorithmus sowie den physikalischen Begrenzungen des Systems erklären.

\subsubsection{Unentschieden durch Blockaden im Endspiel}

In vier Partien endete das Spiel unentschieden: Beide Spieler konnten keine vier Spielsteine mehr in eine Reihe bringen, das Spielfeld war komplett gefüllt. Die Ursache lag dabei nicht in einem technischen Fehler, sondern im Spielverhalten des Roboters. Er verhinderte konsequent alle potenziellen Gewinnchancen des Gegners, agierte aber gleichzeitig zu passiv, um selbst eine entscheidende Reihe aufzubauen.

Der Algorithmus spielte in diesen Situationen eher defensiv und wählte im Mittelspiel häufig sichere, ausgeglichene Positionen, anstatt gezielt einen eigenen Angriff vorzubereiten. Dadurch entstand ein blockiertes Spiel, in dem letztlich keiner der beiden Spieler gewinnen konnte.

\subsubsection{Niederlagen durch fehlende Mehrzugerkennung}

Zwei Partien wurden verloren, weil der Algorithmus eine mehrstufige Kombination des Gegners nicht rechtzeitig erkannte. Ursache ist die eingeschränkte Suchtiefe zu Beginn des Spiels:
\begin{lstlisting}[style=pythonstyle]
	def get_dynamic_depth(board):
		empty = sum(row.count(0) for row in board)
		return 3 if empty > 30 else 4
\end{lstlisting}

In frühen Spielphasen prüft der Algorithmus nur drei Züge voraus, um Rechenzeit zu sparen. Dadurch übersieht er unter Umständen mehrphasige Angriffsstrategien vom Gegner. Ein Spieler nutzte diese Gelegenheit und platzierte seine Spielsteine so, dass der Roboter einen drohenden Vierer erst bemerkte, als keine Abwehr mehr möglich war.


%Der LEGO Spike Hub besitzt mit seinem 100MHz ARM Cortex-M4 Prozessor, 320 KB RAM und 1 MB Flash-Speicher eine stark begrenzte Hardwareausstattung. Diese Ressourcen reichen für einfache Steuerungsaufgaben, setzen aber dem Einsatz komplexer Algorithmen wie Minimax enge Grenzen.

%nsbesondere der geringe Arbeitsspeicher ist entscheidend: Bereits bei einer Suchtiefe von 4 erreicht der Algorithmus die Speichergrenze, da jeder Spielzug rekursiv bewertet und gespeichert wird. Eine tiefere Suche führt zu Speicherüberläufen oder langen Berechnungszeiten.

%Durch Alpha-Beta-Pruning und eine dynamisch begrenzte Suchtiefe lässt sich der Algorithmus dennoch effizient auf dem Hub einsetzen – mit akzeptabler Reaktionszeit und stabiler Ausführung.