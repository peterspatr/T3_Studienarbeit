\usetikzlibrary{circuits.logic.US,circuits.logic.IEC}
      \begin{tikzpicture}[circuit logic US]
      \matrix[column sep=7mm]
      {
      \node (i0) {0}; & & \\
      & \node [and gate] (a1) {}; & \\
      \node (i1) {0}; & & \node [or gate] (o) {};\\
      & \node [nand gate] (a2) {}; & \\
      \node (i2) {1}; & & \\
      };
      \draw (i0.east) -- ++(right:3mm) |- (a1.input 1);
      \draw (i1.east) -- ++(right:3mm) |- (a1.input 2);
      \draw (i1.east) -- ++(right:3mm) |- (a2.input 1);
      \draw (i2.east) -- ++(right:3mm) |- (a2.input 2);
      \draw (a1.output) -- ++(right:3mm) |- (o.input 1);
      \draw (a2.output) -- ++(right:3mm) |- (o.input 2);
      \draw (o.output) -- ++(right:3mm) node [right] {$y$ \quad Hier könnte Ihre Formel $y=(0 \land 0) \lor \overline{( 0 \land 1)}$ stehen};
 \end{tikzpicture}
