\chapter{Spieltheoretische Grundlagen}
In diesem Kapitel werden die grundlegenden Konzepte der Spieltheorie eingeführt, die als mathematische Grundlage für strategische Entscheidungsfindung dienen. Zunächst erfolgt eine Definition der Spieltheorie sowie eine Klassifikation der verschiedenen Spielarten, die für diese Studienarbeit relevant sind. Anschließend werden zentrale spieltheoretische Konzepte erläutert, die sowohl allgemein anwendbar sind als auch speziell auf das Spiel 4Gewinnt zugeschnitten werden.
%Powerpoint zur Spieltheorie:  https://www.mathematik.uni-muenchen.de/~spielth/vortrag/viergewinnt_praesentation.pdf

\section{Definition und Konzepte der Spieltheorie}
Die Spieltheorie ist ein Teilgebiet der Mathematik, das strategische Entscheidungen in Konflikt- oder Kooperationssituationen analysiert. Ein Spiel besteht dabei aus mehreren Elementen: den Spielern, den möglichen Zügen, den Regeln und dem Ergebnis. Ziel der Spieltheorie ist es, optimale Strategien zu entwickeln und die möglichen Ergebnisse eines Spiels vorherzusagen, wenn die Spieler rational handeln.

\textbf{Grundelemente eines Spiels sind folgende Aspekte:}

\begin{enumerate}
	\item \textbf{Spieler:} Die Teilnehmer eines Spiels, die Entscheidungen treffen. Es können zwei oder mehr Mitspieler beteiligt sein.
	\item \textbf{Strategien:} Die Handlungsoptionen, die den Spielern zur Verfügung stehen, um ihre Ziele (Sieg gegen den Mitspieler)zu erreichen.
	\item \textbf{Regeln:} Die Struktur des Spiels legt fest, wie die Züge ablaufen, in welcher Reihenfolge sie gemacht werden und unter welchen Bedingungen das Spiel endet.
	\item \textbf{Auszahlungen:} Die Ergebnisse des Spiels werden für jeden Spieler anhand einer bestimmten Kombination von Strategien festgelegt. Diese Ergebnisse können in Form von Gewinnen, Verlusten oder Nutzen dargestellt werden.
	\item \textbf{Informationen:} Spiele können sich darin unterscheiden, ob die Spieler alle Informationen haben (das heißt, sie sehen alle Züge) oder ob es Unsicherheiten gibt (z.B. verdeckte Karten oder Würfelergebnisse).\autocite{holler_einfuhrung_2019}.
\end{enumerate}

\section{Klassifikation von Spielen}
Um eine Strategie von 4Gewinnt umsetzen zu können, muss das Spiel an sich ersteinmal analysiert werden. In der Spieltheorie werden Spiele nach bestimmten Kriterien eingeteilt, um ihre Struktur, Dynamik und Lösbarkeit besser zu erfassen. Die wichtigsten Kategorien für diese Einordnung sind im Folgenden aufgeführt.
\begin{enumerate}
	\item \textbf{Anzahl der Spieler}
	\begin{itemize}
		\item \textbf{Zwei-Personen-Spiele}: Spiele mit genau zwei Spielern. Solche Spiele werden oft spieltheoretisch analysiert.
		\item \textbf{Mehrspieler-Spiele}: Spiele mit drei oder mehr Teilnehmern. Die Strategien sind komplexer, da Bündnisse und wechselnde Zusammenschlüsse eine Rolle spielen können.
	\end{itemize}
	
	\item \textbf{Spielzüge: gleichzeitige vs. aufeinanderfolgende Entscheidungen}
	\begin{itemize}
		\item \textbf{Simultane Spiele}: Beide Spieler treffen ihre Entscheidungen gleichzeitig, ohne zu wissen, für welche Option der andere sich entschieden hat. Solche Spiele werden oft in einer Matrixform betrachtet, um sie besser zu analysieren.
		\item \textbf{Sequentielle Spiele}: Die Spieler machen ihre Züge nacheinander und haben die Möglichkeit, auf die Aktionen des Gegners zu reagieren. Um die Reihenfolge der Züge und die verschiedenen Entscheidungsmöglichkeiten anschaulich darzustellen, werden solche Spiele oft in Form eines Baumdiagramms, also einer Extensivform, visualisiert.
	\end{itemize}
	
	\item \textbf{Informationen der Spieler}
	\begin{itemize}
		\item \textbf{Spiele mit vollständiger Information}: Alle Spieler sind über die bisherigen Züge und den aktuellen Stand des Spiels informiert. In solchen Situationen lassen sich optimale Strategien durch Rückwärtsinduktion erarbeiten.
		\item \textbf{Spiele mit unvollständiger Information}: Mindestens ein Spieler hat keine vollständigen Informationen über den Spielzustand oder die Strategie des Gegners. Dies führt zu Unsicherheiten bei der Entscheidungsfindung.
	\end{itemize}
	
	\item \textbf{Kooperative vs. nicht-kooperative Spiele}
	\begin{itemize}
		\item \textbf{Kooperative Spiele}: Spieler haben die Möglichkeit, Zusammenschlüsse zu vereinbaren und gemeinsam zu handeln, um sich einen Vorteil zu verschaffen. Dabei spielen Verhandlungen und Absprachen eine entscheidende Rolle für die Ergebnisse.
		\item \textbf{Nicht-kooperative Spiele}: Jeder Spieler agiert unabhängig und verfolgt ausschließlich seine eigenen Ziele.
	\end{itemize}
	
	\item \textbf{Gewinnstruktur: Nullsummenspiele vs. Nicht-Nullsummenspiele}
	\begin{itemize}
		\item \textbf{Nullsummenspiele}: Der Gewinn eines Spielers ist genau der Verlust des anderen. Diese Art von Spielen ist konfliktbeladen, weil der Vorteil für einen Spieler immer mit einem Nachteil für den anderen einhergeht.
		\item \textbf{Nicht-Nullsummenspiele}: Die Gewinne und Verluste der Spieler müssen nicht unbedingt übereinstimmen. In solchen Spielen kann es vorkommen, dass die Teilnehmer zusammenarbeiten oder ihre Gewinne teilen.
	\end{itemize}
	
	\item \textbf{Deterministische vs. stochastische Spiele}
	\begin{itemize}
		\item \textbf{Deterministische Spiele}: Es gibt keine Zufälle im Spiel, alles hängt ganz von den Entscheidungen der Spieler ab.
		\item \textbf{Stochastische Spiele}: Zufallselemente wie Würfel oder Karten bringen unvorhersehbare Wendungen ins Spiel, was für zusätzliche Spannungen und Unsicherheiten sorgt.
	\end{itemize}
	
	\item \textbf{Endliche vs. unendliche Spiele}
	\begin{itemize}
		\item \textbf{Endliche Spiele}: Die Anzahl der möglichen Spielzustände und Züge ist begrenzt. Das bedeutet, dass man solche Spiele komplett durchleuchten und die besten Strategien finden kann.
		\item \textbf{Unendliche Spiele}: Spiele, bei denen es potenziell unendlich viele Züge oder Zustände gibt und die sich nicht vollständig analysieren lassen \autocites{Winter2019}.
	\end{itemize}
\end{enumerate}

\subsection*{Kriterien angewandt auf 4Gewinnt}
Anhand diesen genannten Kriterien kann 4Gewinnt klassifiziert werden. 4Gewinnt ist nach den aufgelisteten fünf Kriterien ein kombinatorisches Spiel. Sie bieten einen idealen Rahmen, um Strategien, Entscheidungsfindung und Spieltheorie zu untersuchen. Auch weitere Spiele wie Schach, Dame, Mühle und Tic-Tac-Toe sind ebenfalls kombinatorische Spiele. \\

\textbf{Diese folgende fünf Eigenschaften machen ein kombinatorisches Spiel aus:}
\begin{enumerate}
	\item   \textbf{Zwei-Personen-Spiel: }4Gewinnt ist ein klassisches Spiel für zwei Personen, bei dem die Spieler abwechselnd ihre Steine in ein Gitter fallen lassen. 
	\item 	\textbf{Nullsummenspiel: }In "Vier Gewinnt" verfolgen die beiden Spieler gegensätzliche Ziele. Wenn einer von ihnen gewinnt, indem er vier Steine in einer Reihe hat, bedeutet das gleichzeitig, dass der andere verloren hat.
	\item 	\textbf{endliches Spiel: } Das Spielfeld hat insgesamt 42 Felder, die sich auf 6 Reihen und 7 Spalten verteilen. Nach einer bestimmten Anzahl an Zügen wird das Feld immer vollständig gefüllt sein. Das bedeutet, dass jedes Spiel immer zu einem Ende kommt – entweder gewinnt ein Spieler oder es endet unentschieden, wenn keine Züge mehr möglich sind.
	\item 	\textbf{vollständige Information: } Beide Spieler können jederzeit den aktuellen Spielstand genau im Blick behalten, da alle Spielsteine offen liegen. Es gibt keine geheimen Informationen, sodass sie ihre Strategien auf Grundlage aller bisherigen Züge entwickeln können.
	\item 	\textbf{Determinismus:} Es gibt keine Zufälle, der Ausgang des Spiels hängt ganz von den Entscheidungen der Spieler ab. Jede Spielsituation ist klar und vorhersehbar, solange man die Züge der Spieler kennt.
\end{enumerate}

Durch diese genannten fünf Eigenschaften werden kombinatorische Spiele auch \textbf{endliches ZweiPersonen-Nullsummenspiel mit perfekter Information} genannt\autocite{holler_einfuhrung_2019}.

\section{Relevante spieltheoretische Konzepte für 4Gewinnt}
Mit Kapitel 3.2 wurde dargestellt, dass das 4Gewinnt ein strategisches Zwei-Personen-Spiel mit vollständiger Information ist und somit mithilfe der Spieltheorie analysieren lässt. Einige zentrale Konzepte der Spieltheorie sind besonders relevant, um die Dynamik und Strategien von 4Gewinnt zu verstehen:


\textbf{Dominante Strategien:}
\begin{itemize}
	\item Eine Strategie ist dominant, wenn sie unabhängig von den Zügen des Gegners optimal ist.
	\item In 4Gewinnt existieren situationsabhängige dominante Strategien.
	\item Die Kontrolle der Mittelspalte gilt als teilweise dominante Strategie, da sie zentrale taktische Vorteile bietet.
\end{itemize}

\textbf{Gleichgewichtskonzepte:}
\begin{itemize}
	\item \textbf{Nash-Gleichgewicht}: Ein Zustand, in dem kein Spieler durch einseitiges Abweichen ihre Position verbessern kann.
	\item \textbf{Teilspielperfekte Gleichgewichte}: Strategien, die in allen Teilspielen optimal sind.
	\item \textbf{Minimax-Prinzip}: Eine Methode, bei der der Spieler versucht, den maximal möglichen Verlust zu minimieren.
\end{itemize}

\textbf{Strategische Tiefe:}
\begin{itemize}
	\item mehrere Züge im Voraus denken ist entscheidend für den Erfolg.
	\item Erkennung und gezielte Nutzung von Zwangszügen und Zugfolgen.
	\item Entwicklung von Bedrohungspotenzialen und deren effektive Abwehr.
\end{itemize}

\textbf{Positionsbewertung:}
\begin{itemize}
	\item Bewertung von Spielstellungen nach strategischen Kriterien zur Einschätzung der Spielstärke.
	\item Erkennung von entscheidenden Momenten, die das Ergebnis des Spiels beeinflussen können.
	\item Untersuchung von Gewinnmöglichkeiten und Risiken, um potenzielle Spielstrategien zu entwerfen.
\end{itemize}

\textbf{Zugzwang-Situationen:}
\begin{itemize}
	\item Zugzwang beschreibt Situationen, in denen jeder mögliche Zug die eigene Position verschlechtert.
	\item Solche Situationen haben eine hohe strategische Bedeutung für die Spielführung.
	\item Es ist wichtig, gezielte Methoden anzuwenden, um den Gegner in eine schwierige Situation zu bringen und ihn unter Druck zu setzen.
\end{itemize}

Basierend auf den vorgestellten Konzepte kann nun eine Strategie erstellt werden, der dabei hilft, einen Algorithmus zu konzipieren. Dieser Algorithmus würde verschiedene Elemente der Spieltheorie und strategischen Überlegungen einbeziehen, um die Entscheidungsfindung zu verbessern \autocites{monien_alphabeta-algorithmus_2008}.