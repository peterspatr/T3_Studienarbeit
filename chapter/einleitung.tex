\chapter{Einleitung}
\label{cha:Einleitung}

%Folgende Stichworte können zum Aufbau der Einleitung herangezogen werden.
%
%\begin{itemize}
%\item Hinführung, Begründung, Zweck und Ziel der Aufgabenstellung
%\item Erläuterung der Problemstellung
%\item Konkretisierung der zu lösenden Aufgabe
%\item Gegebenenfalls Formulierung einer Leitfrage oder Forschungsfrage
%\item Ausgangslage, geplante Vorgehensweise, Methoden zur Bearbeitung und Zielsituation
%\item Zum Ende der Einleitung wird eine Kurzübersicht über die Inhalte der Kapitel gegeben: \glqq Die Arbeit ist wie folgt gegliedert: ...\grqq
%\end{itemize}
%
%Die Einleitung wird üblicherweise auf ein bis zwei Seiten als fortlaufender Text geschrieben. Eine weitere Untergliederung in nummerierte Abschnitte ist nicht empfehlenswert, da dies erstens unüblich ist, zweitens die Lesbarkeit nicht begünstigt und drittens die Formulierung der Einleitung erschwert. Weitere Empfehlungen zum Aufbau der Einleitung und des gesamten Dokuments sind z.~B. aus \autocite{DHBW.2021} und \autocite{Lindenlauf.2022} zu entnehmen.
%
%\clearpage
%
%Hinweise: 
%
%\begin{itemize}
%\item Auch in der Einleitung unbedingt zu wichtigen Hintergründen und Fakten Zitate aufführen.\index{Zitat} Zitate bitte in der Form \autocite{Tipler.2019} oder mit Seitenbezug \autocite[66]{Ziegler.2017} oder auch mehrere Zitate  \autocite{Tipler.2019, Ziegler.2017} innerhalb einer eckigen Klammer angeben. Zur besseren Lesbarkeit bitte immer ein Leerzeichen vor dem Zitat einfügen.
%	
%\item Bereits in der Einleitung können Abkürzungen erläutert werden. Grundsätzlich gilt, dass bei der ersten Verwendung einer Abkürzung diese auch erläutert wird. Zum Beispiel können das Antiblockiersystem (ABS) oder die Fahrdynamikregelung (Electronic Stability Control, ESC) als Abkürzungen eingeführt werden. In der Datei \textit{pages/abkuerzungen.tex} sind alle verwendeten Abkürzungen einzufügen. Neben dem verpflichtenden Abkürzungsverzeichnis kann auch ein Glossar hinzugefügt werden. In dieser Vorlage können Glossareinträge in der Datei \textit{pages/glossar.tex} eingefügt werden. Ein \Gls{gls:glossar} ist jedoch nicht verpflichtend.
%\end{itemize}



\textcolor{red}{*********************Muss noch Korrektur gelesen werden******************************************}
Die Entwicklung eines Vier-Gewinnt-Roboters mit der Verwendung des LEGO Spike Prime Systems stellt eine faszinierende Herausforderung an der Schnittstelle von Spieltheorie, Informatik und Robotik dar. Diese Studienarbeit befasst sich mit der systematischen Konzeptionierung und Umsetzung eines solchen Roboters. Er soll in der Lage sein, autonom und effektiv als Gegner Vier Gewinnt zu spielen. \\
Vier Gewinnt, ein klassisches und beliebtes Strategiespiel für zwei Personen, bietet trotz seiner scheinbaren Einfachheit eine sehr große strategische Tiefe.
Die Aufgabe, einen Roboter zu entwickeln, der dieses Spiel beherrscht, erfordert nicht nur ein tiefgreifendes Verständnis der Spielmechanik. Sondern es erfordert auch die Fähigkeit, komplexe Algorithmen in einem  Umfeld  mit beschränkten Ressourcen zu entwerfen.

Im Rahmen des ersten Teils dieser Arbeit werden zunächst die spieltheoretischen Grundlagen von Vier Gewinnt näher eingegangen. Dabei werden zentrale Konzepte wie dominante Strategien, Nash-Gleichgewichte und das Minimax-Prinzip auf Vier Gewinnt angewandt und näher analysiert.\\
Ein besonderer Fokus wird dabei auf die Entwicklung und Optimierung von Algorithmen gelegt. Insbesondere auf den Alpha-Beta-Algorithmus. Dieser ermöglicht es, effizient optimale Spielzüge zu berechnen.

Am Ende dieser wird ein grober Systementwurf des Roboters dargestellt. Die Herausforderung dabei besteht darin, die theoretischen Erkenntnisse in ein funktionierendes System zu überführen. Es sollte in der Lage sein, Spielsituationen zu erfassen und Entscheidungen zu treffen und dadurch präzise Spielzüge auszuführen zu können.
Diese Studienarbeit zielt darauf ab, nicht nur einen funktionsfähigen Vier-Gewinnt-Roboter zu entwickeln, sondern auch die Verbindung zwischen theoretischen Konzepten und praktischer Anwendung in der Robotik zu demonstrieren. Sie wird dabei in folgende Unterpunkte eingeteilt.

\newpage

\begin{itemize}
	\item \textbf{Grundlagen des Spiels Vier Gewinnt:} In diesem Kapitel geht es um die Erläuterung des Spielverlaufs und der Spielregeln. Ebenso wird auch auf den historischen Hintergrund sowie auf die mathematischen Eigenschaften von Vier Gewinnt eingegangen.

\item \textbf{Spieltheoretische Grundlagen:} Im zweiten Kapitel geht es um Definition und Konzepte der Spieltheorie. Ebenso um die Klassifikation von Spielen und relevante spieltheoretische Konzepte für Vier Gewinnt.

\item \textbf{Analyse von Vier Gewinnt aus spieltheoretischer Sicht:} Im Kapitel hier wird auf die Darstellung des Spiels in Normalform, Darstellung des Spiels in Extensivform, Komplexität des Spielbaums und Strategien und Gleichgewichte eingegangen.

\item \textbf{Lösungsansätze für Vier Gewinnt:} In Kapitel fünf werden optimale Strategien
Heuristische Ansätze analysiert. Es wird aber auch auf die Entwicklung von Algorithmen (MinMax und AlphaBeta) eingegangen.

\item \textbf{Anwendung der Spieltheorie auf Vier Gewinnt:} Hierbei wird auf die Analyse von Spielsituationen, ebenso auf die Berechnung von Gewinnwahrscheinlichkeiten und die Entwicklung von Spielstrategien näher eingegangen. 

\item \textbf{Systementwurf:} Das letzte Kapitel befasst sich mit den Komponenten für die Realisierung der Hardware. Es wird aber auch auf das Zusammenspiel von Soft- und Hardware näher eingegangen.
\end{itemize}




