\chapter{Einleitung}
\label{cha:Einleitung}

%Folgende Stichworte können zum Aufbau der Einleitung herangezogen werden.
%
%\begin{itemize}
%\item Hinführung, Begründung, Zweck und Ziel der Aufgabenstellung
%\item Erläuterung der Problemstellung
%\item Konkretisierung der zu lösenden Aufgabe
%\item Gegebenenfalls Formulierung einer Leitfrage oder Forschungsfrage
%\item Ausgangslage, geplante Vorgehensweise, Methoden zur Bearbeitung und Zielsituation
%\item Zum Ende der Einleitung wird eine Kurzübersicht über die Inhalte der Kapitel gegeben: \glqq Die Arbeit ist wie folgt gegliedert: ...\grqq
%\end{itemize}
%
%Die Einleitung wird üblicherweise auf ein bis zwei Seiten als fortlaufender Text geschrieben. Eine weitere Untergliederung in nummerierte Abschnitte ist nicht empfehlenswert, da dies erstens unüblich ist, zweitens die Lesbarkeit nicht begünstigt und drittens die Formulierung der Einleitung erschwert. Weitere Empfehlungen zum Aufbau der Einleitung und des gesamten Dokuments sind z.~B. aus \autocite{DHBW.2021} und \autocite{Lindenlauf.2022} zu entnehmen.
%
%\clearpage
%
%Hinweise: 
%
%\begin{itemize}
%\item Auch in der Einleitung unbedingt zu wichtigen Hintergründen und Fakten Zitate aufführen.\index{Zitat} Zitate bitte in der Form \autocite{Tipler.2019} oder mit Seitenbezug \autocite[66]{Ziegler.2017} oder auch mehrere Zitate  \autocite{Tipler.2019, Ziegler.2017} innerhalb einer eckigen Klammer angeben. Zur besseren Lesbarkeit bitte immer ein Leerzeichen vor dem Zitat einfügen.
%	
%\item Bereits in der Einleitung können Abkürzungen erläutert werden. Grundsätzlich gilt, dass bei der ersten Verwendung einer Abkürzung diese auch erläutert wird. Zum Beispiel können das Antiblockiersystem (ABS) oder die Fahrdynamikregelung (Electronic Stability Control, ESC) als Abkürzungen eingeführt werden. In der Datei \textit{pages/abkuerzungen.tex} sind alle verwendeten Abkürzungen einzufügen. Neben dem verpflichtenden Abkürzungsverzeichnis kann auch ein Glossar hinzugefügt werden. In dieser Vorlage können Glossareinträge in der Datei \textit{pages/glossar.tex} eingefügt werden. Ein \Gls{gls:glossar} ist jedoch nicht verpflichtend.
%\end{itemize}



****************************Muss noch geändert werden****************************************************
Die Entwicklung eines Vier-Gewinnt-Roboters mit der Verwendung des LEGO Spike Prime Systems stellt eine faszinierende Herausforderung an der Schnittstelle von Spieltheorie, Informatik und Robotik dar. Diese Studienarbeit befasst sich mit der systematischen Konzeptionierung und Umsetzung eines solchen Roboters. Er soll in der Lage sein, autonom und effektiv als Gegner Vier Gewinnt zu spielen.\\
Vier Gewinnt, ein klassisches und beliebtes Strategiespiel für zwei Personen, bietet trotz seiner scheinbaren Einfachheit eine beachtliche strategische Tiefe.
Die Aufgabe, einen Roboter zu entwickeln, der dieses Spiel beherrscht, erfordert nicht nur ein tiefgreifendes Verständnis der Spielmechanik, sondern auch die Fähigkeit, komplexe Algorithmen in einem ressourcenbeschränkten Umfeld zu implementieren.\\

Im Rahmen dieser Arbeit werden zunächst die spieltheoretischen Grundlagen von Vier Gewinnt analysiert. Dabei werden zentrale Konzepte wie dominante Strategien, Nash-Gleichgewichte und das Minimax-Prinzip auf das Spiel angewandt. Ein besonderer Fokus liegt auf der Entwicklung und Optimierung von Algorithmen, insbesondere dem Alpha-Beta-Algorithmus, der es ermöglicht, effizient optimale Spielzüge zu berechnen.
Die praktische Umsetzung erfolgt mit dem LEGO Spike Prime System, das eine flexible Plattform für die Roboterkonstruktion und -programmierung bietet. Die Herausforderung besteht darin, die theoretischen Erkenntnisse in ein funktionierendes System zu überführen, das in der Lage ist, Spielsituationen zu erfassen, Entscheidungen zu treffen und präzise Spielzüge auszuführen.
Diese Studienarbeit zielt darauf ab, nicht nur einen funktionsfähigen Vier-Gewinnt-Roboter zu entwickeln, sondern auch die Verbindung zwischen theoretischen Konzepten und praktischer Anwendung in der Robotik zu demonstrieren. Sie bietet damit einen umfassenden Einblick in die interdisziplinäre Natur moderner technischer Herausforderungen.






