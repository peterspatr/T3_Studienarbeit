\chapter{Grundlagen des Spiels Vier Gewinnt}

\section{Spielregeln und Spielablauf}
Die Regeln für dieses Strategiespiel sind kinderleicht, was auch ein Argument für die große Beliebtheit bei Jung und Alt ist. Das Spiel besteht aus einem senkrecht stehenden Spielbrett und jeweils aus 21 gelben und roten runde Steine. Diese Steine werden abwechselnd in das hohle Spielbrett mit 42 Aussparungen, sieben Spalten und sechs Reihen, eingeworfen. Der Spieler kann somit eine Spalte auswählen und den Stein fallen lassen. Der eingeworfene Stein besetzt den untersten freien Platz dieser Spalte. Das Spiel wird dann gewonnen, wenn einer der Spieler vier oder mehr Steine in einer waagerechten, senkrechten oder diagonalen Reihe platzieren kann. Der andere Spieler hat somit automatisch verloren. Kommt es zu keiner Bildung einer dieser Kombinationen, endet das Spiel bei Vergabe aller Steine in einem Remis.

\section{Historischer Hintergrund}
\section{Mathematische Eigenschaften des Spiels}
