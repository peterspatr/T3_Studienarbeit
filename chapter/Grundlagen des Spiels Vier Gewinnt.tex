\chapter{Grundlagen des Spiels Vier Gewinnt}

\section{Spielregeln und Spielablauf}
%Die Regeln für dieses Strategiespiel sind kinderleicht, was auch ein Argument für die große Beliebtheit bei Jung und Alt ist. Das Spiel besteht aus einem senkrecht stehenden Spielbrett und jeweils aus 21 gelben und roten runde Steine. Diese Steine werden abwechselnd in das hohle Spielbrett mit 42 Aussparungen, sieben Spalten und sechs Reihen, eingeworfen. Der Spieler kann somit eine Spalte auswählen und den Stein fallen lassen. Der eingeworfene Stein besetzt den untersten freien Platz dieser Spalte. Das Spiel wird dann gewonnen, wenn einer der Spieler vier oder mehr Steine in einer waagerechten, senkrechten oder diagonalen Reihe platzieren kann. Der andere Spieler hat somit automatisch verloren. Kommt es zu keiner Bildung einer dieser Kombinationen, endet das Spiel bei Vergabe aller Steine in einem Remis.

Das originale Spiel 4Gewinnt besteht aus einer Rasterwand mit sechs Zeilen und sieben Spalten, also 42 Löchern. Außerdem besteht es aus 21 roten Spielchips und 21 gelben Spielchips. 4Gewinnt lässt sich nur zu zweit spielen. Ziel des Spieles ist es, vier Spielchips einer Farbe in eine Reihe (waagrecht, senkrecht oder diagonal) zu bringen. Der jüngste Spieler beginnt das Spiel. Der Spieler, der an der Reihe ist, wirft einen Spielstein seiner Farbe durch die Öffnung der Rasterwand, wo durch der gespielte Stein auf den untersten freien Platz in der Spalte fällt. Danach ist der andere Spieler an der Reihe. Die Spieler werfen so lange ihre Spielchips in das Raster, bis einer das Ziel erreicht hat oder alle 42 Felder belegt sind. Für den Fall, dass alle 42 Felder des Rasters belegt sind, endet das Spiel unentschieden. Um ein neues Spiel zu starten, muss ein Spieler die Steine aus dem Raster fallen lassen. Diese werden dann wieder unter den Spielern aufgeteilt und eine neue Runde 4Gewinnt kann beginnen\autocite{Hasbro.2020}.

\begin{figure}[H]
	\centering
	\includegraphics[width=0.8\linewidth]{images/Diagonal}
	\caption[Vier rote Steine diagonal]{Vier rote Steine diagonal im Raster}
	\label{fig:diagonal}
\end{figure}
\begin{figure}[H]
	\centering
	\includegraphics[width=0.8\linewidth]{images/Senkrecht}
	\caption[Vier rote Steine in Reihe senkrecht]{Vier rote Steine senkrecht im Raster}
	\label{fig:senkrecht}
\end{figure}
\begin{figure}[H]
	\centering
	\includegraphics[width=0.8\linewidth]{images/Waagrecht}
	\caption[Vier rote Steine in Reihe waagrecht]{Vier rote Steine waagrecht im Raster}
	\label{fig:waagrecht}
\end{figure}
\begin{figure}[H]
	\centering
	\includegraphics[width=0.8\linewidth]{images/Unentschieden}
	\caption[Unentschieden]{Unentschieden - es kam keine vierer Reihe zustande}
	\label{fig:unentschiedent}
\end{figure}
\newpage
\section{Historischer Hintergrund}
Das Spiel “Vier Gewinnt” wurde 1973 von Howard Wexler und Ned Strongin entwickelt. Die Erstveröffentlichung erfolgte 1974 durch die Firma Milton Bradley, die im deutschsprachigen Raum als MB Spiele bekannt ist.
Die Entwicklung des Spiels erfolgte durch die Strongin und Wexler Corporation in den USA. Obwohl der Preis der Erstausgabe nicht überliefert ist, wurde das Spiel schnell zu einem beliebten Strategiespiel.\\
Eine interessante Entwicklung erfolgte bereits vor der Entstehung des klassischen Vier Gewinnt: 1967 erschien in den USA eine dreidimensionale Variante namens “Score Four”. Diese wurde 1974 in Deutschland von Ravensburger unter dem Namen “Sogo” vertrieben und war in der DDR als “Raummühle” bekannt.\\
Das klassische Vier Gewinnt etablierte sich rasch als beliebtes Familienspiel und wurde über die Jahre in verschiedenen Varianten neu aufgelegt. Die Einfachheit der Regeln bei gleichzeitiger strategischer Tiefe trug maßgeblich zum anhaltenden Erfolg des Spiels bei.\\
https://www.gameorama.ch/de/museum/spielmuseum/timeline/gewinnt?t
\section{Mathematische Eigenschaften des Spiels}
Das Spiel 4Gewinnt weist mehrere wichtige mathematische Eigenschaften auf, die für seine Analyse fundamental sind.\\
Hierbei liegt vorallem der Fokus auf Spielfeldgröße und Kombinatorik. Das klassische Spielfeld besteht aus 7 Spalten und 6 Reihen, was 42 mögliche Positionen ergibt. Diese Dimension wurde damals bewusst festgelegt, um ein ausgewogenes Verhältnis zwischen Komplexität und Spielbarkeit zu gewährleisten.\\

Die Gesamtzahl der theoretisch möglichen Spielzustände lässt sich wie folgt berechnen:
\begin{itemize}
	\item Jede Position kann drei Zustände annehmen: leer, Spieler 1, Spieler 2
	\item Dies ergibt theoretisch  mögliche Zustände\\
	
	Die tatsächliche Anzahl ist jedoch deutlich geringer, da:
	\item Spielsteine nur von unten nach oben gesetzt werden können
	\item Das Spiel endet, sobald vier Steine in einer Reihe liegen
	\item Nicht alle Kombinationen sind durch legale Spielzüge erreichbar
\end{itemize}

Komplexitätsgrad
•	Der durchschnittliche Verzweigungsfaktor (mögliche Züge pro Spielsituation) liegt bei etwa 4
•	Die maximale Spieltiefe beträgt 42 Züge
•	Die Spielbaumkomplexität ist deutlich geringer als bei Schach, aber höher als bei Tic-Tac-Toe
Symmetrieeigenschaften
Das Spielfeld weist eine vertikale Symmetrie auf, wodurch sich die Anzahl der zu analysierenden Positionen reduziert. Die mittlere Spalte nimmt dabei eine besondere strategische Position ein.
Diese mathematischen Eigenschaften bilden die Grundlage für die spieltheoretische Analyse und die Entwicklung von Lösungsstrategien.