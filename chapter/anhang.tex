\addchap{A Nutzung von Künstliche Intelligenz basierten Werkzeugen}
\setcounter{chapter}{1}

Im Rahmen dieser Arbeit wurden Künstliche Intelligenz (KI)\index{Künstliche Intelligenz} basierte Werkzeuge benutzt. Tabelle~\ref{tab:anhang_uebersicht_KI_werkzeuge} gibt eine Übersicht über die verwendeten Werkzeuge und den jeweiligen Einsatzzweck.

\begin{table}[hbt]	
	\centering
	\renewcommand{\arraystretch}{1.5}	% Skaliert die Zeilenhöhe der Tabelle
	\captionabove[Liste der verwendeten Künstliche Intelligenz basierten Werkzeuge]{Liste der verwendeten KI basierten Werkzeuge}
	\label{tab:anhang_uebersicht_KI_werkzeuge}
	\begin{tabular}{>{\raggedright\arraybackslash}p{0.3\linewidth} >{\raggedright\arraybackslash}p{0.65\linewidth}}
		\textbf{Werkzeug} & \textbf{Beschreibung der Nutzung}\\
		\hline 
		\hline
		ChatGPT & 	\vspace{-\topsep}
		\begin{itemize}[noitemsep,topsep=0pt,partopsep=0pt,parsep=0pt] 
			\item Grundlagenrecherche zu bekannten Prinzipien optischer Sensorik zur Abstandsmessung (siehe Abschnitt \ldots)
			\item Suche nach Herstellern von Lidar-Sensoren (siehe Abschnitt \ldots)
			\item \ldots
		\end{itemize} \\
		ChatPDF &	\vspace{-\topsep}
		\begin{itemize}[noitemsep,topsep=0pt,partopsep=0pt,parsep=0pt] 
			\item Recherche und Zusammenfassung von wissenschaftlichen Studien im Themenfeld \ldots
			\item \ldots
		\end{itemize} \\ 
		DeepL	&	\vspace{-\topsep}
		\begin{itemize}[noitemsep,topsep=0pt,partopsep=0pt,parsep=0pt] 
			\item Übersetzung des Papers von $[\ldots]$
		\end{itemize} \\ 
		Tabnine AI coding assistant &	\vspace{-\topsep}
		\begin{itemize}[noitemsep,topsep=0pt,partopsep=0pt,parsep=0pt] 
			\item Aktiviertes Plugin in MS Visual Studio zum Programmieren des \ldots
			\item \ldots
		\end{itemize} \\ 
		\ldots	&	\vspace{-\topsep}
		\begin{itemize}[noitemsep,topsep=0pt,partopsep=0pt,parsep=0pt] 
			\item \ldots
		\end{itemize} \\ 
		\hline 
	\end{tabular} 
\end{table}