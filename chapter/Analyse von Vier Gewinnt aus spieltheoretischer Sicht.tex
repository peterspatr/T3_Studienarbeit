\chapter{Analyse von Vier Gewinnt aus spieltheoretischer Sicht}

\section{Darstellung des Spiels in Normalform und Extensivform}
Die spieltheoretische Analyse von Vier Gewinnt erfordert eine formale mathematische Darstellung des Spiels. Dabei sind zwei grundlegende Darstellungsformen von besonderer Bedeutung:

Die Extensivform stellt das Spiel als Spielbaum dar und wird durch den Tupel  beschrieben T= (N,K,P,U,h,p), wobei:
\begin{itemize}
	\item N die Menge der Spieler (Spieler 1 und 2)
	\item K die Menge aller Knoten (Spielpositionen)
	\item P die Menge der Spielerpositionen
	\item U die Menge der Endknoten
	\item h die Auszahlungsfunktion
	\item p die Vorgängerfunktion ist
\end{itemize}
Für Vier Gewinnt ist die Extensivform besonders geeignet, da sie die sequentielle Struktur des Spiels abbildet, die perfekte Information widerspiegelt und eine intuitive Analyse von Zugfolgen ermöglicht.

Die Normalform hingegen repräsentiert das Spiel als Matrix G=(N,S,u) , mit:
\begin{itemize}
	\item N als Menge der Spieler
	\item S als Menge der Strategieprofile
	\item u als Auszahlungsfunktion
\end{itemize}

Die mathematische Beschreibung des Spielbaums erfolgt durch:
T = (V,E,v_0) 
wobei:
\begin{itemize}
	\item V die Menge aller Knoten
	\item E die Menge der Kanten (mögliche Züge)
	\item v_0 der Wurzelknoten (Ausgangsposition) ist
\end{itemize}

Die Darstellung in Normalform ist aufgrund der hohen Anzahl möglicher Strategien sehr umfangreich und für praktische Analysen weniger geeignet. Die Größe der Normalform-Matrix wächst exponentiell mit der Spieltiefe.

\section{Strategien und Gleichgewichte}
\section{Komplexität des Spielbaums}
	