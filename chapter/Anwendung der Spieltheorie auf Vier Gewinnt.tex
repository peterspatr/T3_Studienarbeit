\chapter{Anwendung der Spieltheorie auf Vier Gewinnt}
\label{cha:nwendung der Spieltheorie auf Vier Gewinnt}
In diesem Kapitel werden die spieltheoretischen Aspekte von Vier Gewinnt, einschließlich der Analyse von Spielsituationen, der Berechnung von Gewinnwahrscheinlichkeiten und der Entwicklung von Spielstrategien untersucht.


\section{Analyse von Spielsituationen}
\label{sec:Analyse von Spielsituationen}
Für das Verständnis der Spieldynamik und die Entwicklung effektiver Strategien, ist die Analyse der Spielsituation entscheidend.
Vier Gewinnt, ist ein zwei Personen Spiel mit vollständiger Information. Das bedeutet, alle Spieler zu jeder Zeit den gesamten Zustand des Spiels kennen \autocite{ruile2009viergewinnt}.

Der Bestimmungssatz von Ernst Zermelo besagt, dass sich jedes kombinatorische Spiel in eine der folgenden Kategorien einordnen lässt.
\begin{enumerate}
	\item  Der beginnende Spieler hat eine dominante Strategie. Wenn das Spiel optimal, ohne Fehler verläuft, gewinnt er die Partie.
	\item Der nachziehende Spieler 2 hat eine dominante Strategie. Wenn das Spiel optimal verläuft, gewinnt dieser.
	\item Keiner der beiden Parteien hat eine dominante Strategie. Wenn beide Spieler optimal spielen, endet das Spiel ohne einen Sieger. Sobald ein Spieler einen Fehler macht, gewinnt der andere Spieler das Spiel \autocite{mueller_2011}.
\end{enumerate}

Aus der Analyse geht hervor, dass bestimmte Situationen von besonderer Bedeutung für einen Sieg sind.
Das bilden einer horizontalen Viererreihe sind schwer zu erreichen. Umso wichtige ist es einen Sieg die Spalten 1-5 zu kontrollieren. 


\section{Berechnung von Gewinnwahrscheinlichkeiten}
Aufgrund der Komplexität des Spiels stellt die Berechnung der Gewinnwahrscheinlichkeit in Anwendung eine Herausforderung dar. Das Spielraster besteht aus sieben Spalten und sechs Zeilen, dadurch entsteht eine enorm große Anzahl an möglichen Zuständen. Was die Wahrscheinlichkeitsberechnung für den Gewinn des Spiels so komplex macht.
Moderne Ansätze zur Berechnung der Gewinnwahrscheinlichkeit greifen auf KI-Technik zurück. Die exakte Berechnung für alle möglichen Spielsituationen bleibt aufgrund der Komplexität von Vier Gewinnt eine Herausforderung \autocite{ruile2009viergewinnt}.
\section{Entwicklung von Spielstrategien}
	