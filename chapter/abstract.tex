\chapter*{Kurzfassung} %*-Variante sorgt dafür, das Abstract nicht im Inhaltsverzeichnis auftaucht

In der vorliegenden Studienarbeit wird die Entwicklung eines Roboters für das Spiel „Vier gewinnt“ unter Verwendung des LEGO Spike Prime Systems behandelt. Ziel des Projekts ist es, einen Roboter zu entwerfen, der in der Lage ist, autonom die Rolle eines menschlichen Gegners im Spiel „Vier gewinnt“ zu übernehmen. Der Roboter soll nicht nur die Spielzüge des menschlichen Mitspielers erkennen und darauf reagieren, sondern auch selbstständig Spielsteine in das Spielfeld einwerfen und sicherstellen, dass das Spielfeld für den nächsten Zug bereit ist. Dies erfordert eine präzise Steuerung des Roboters, insbesondere beim Platzieren der Spielsteine, sowie die Fähigkeit, das Spielfeld zu überwachen, um die Position der bereits gesetzten Steine zu erkennen.
Das Projekt umfasst verschiedene technische und organisatorische Aspekte. Dazu gehört die mechanische Konstruktion des Roboters, einschließlich des Mechanismus zum Einwerfen der Spielsteine und die Überwachung des Spielfelds, sowie die Entwicklung der Software, die die Spielzüge und das Verhalten des Roboters steuert. Ein weiterer wichtiger Punkt ist die Sensorik: Der Roboter muss in der Lage sein, die aktuelle Spielsituation durch Farbsensoren zu erfassen, um zu wissen, welche Felder im Spielfeld bereits besetzt sind und wo er seinen nächsten Spielstein platzieren kann.
Zum Abschluss des Projekts wird ein kleines Turnier organisiert, bei dem die von verschiedenen Teams entwickelten Roboterlösungen gegeneinander antreten. Dies bietet die Möglichkeit, den Roboter unter realen Bedingungen zu testen und seine Fähigkeiten im direkten Vergleich mit den Lösungen der anderen Kommilitonen zu messen. Um zusätzliche Motivation zu schaffen, wird die beste Lösung am Ende prämiert, was den Wettbewerbsgedanken fördert und den Anreiz erhöht eine effektive Lösungen zu entwickeln.

\clearpage

\chapter*{Abstract} %*-Variante sorgt dafür, das Abstract nicht im Inhaltsverzeichnis auftaucht
\label{cha:Abstract}
This student research project deals with the development of a robot for the game “Four Wins” using the LEGO Spike Prime system. The aim of the project is to design a robot that is capable of autonomously assuming the role of a human opponent in the game “Four Wins”. The robot should not only recognize the moves of the human opponent and react to them, but also independently place pieces on the playing field and ensure that the playing field is ready for the next move. This requires precise control of the robot, especially when placing the pieces, as well as the ability to monitor the playing field in order to recognize the position of the pieces already placed.
The project involves various technical and organizational aspects. These include the mechanical design of the robot, including the mechanism for inserting the tiles and monitoring the playing field, as well as the development of the software that controls the moves and behavior of the robot. Another important point is the sensor technology: the robot must be able to detect the current game situation using color sensors in order to know which squares in the playing field are already occupied and where it can place its next piece.
At the end of the project, a small tournament is organized in which the robot solutions developed by different teams compete against each other. This offers

Translated with DeepL.com (free version)
\clearpage