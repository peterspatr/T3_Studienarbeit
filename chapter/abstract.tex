\chapter*{Kurzfassung} %*-Variante sorgt dafür, das Abstract nicht im Inhaltsverzeichnis auftaucht

Problemstellung

Ziel der Arbeit

Vorgehen und angewandte Methoden

Konkrete Ergebnisse der Arbeit, am besten mit quantitativen Angaben

In der vorliegenden Studienarbeit wird die Entwicklung eines Roboters für das Spiel „Vier gewinnt“ unter Verwendung des LEGO Spike Prime Systems behandelt. Ziel des Projekts ist es, einen Roboter zu entwerfen, der in der Lage ist, autonom die Rolle eines menschlichen Gegners im Spiel „Vier gewinnt“ zu übernehmen. Der Roboter soll nicht nur die Spielzüge des menschlichen Mitspielers erkennen und darauf reagieren, sondern auch selbstständig Spielsteine in das Spielfeld einwerfen und sicherstellen, dass das Spielfeld für den nächsten Zug bereit ist. Dies erfordert eine präzise Steuerung des Roboters, insbesondere beim Platzieren der Spielsteine, sowie die Fähigkeit, das Spielfeld zu überwachen, um die Position der bereits gesetzten Steine zu erkennen.
Das Projekt umfasst verschiedene technische und organisatorische Aspekte. Dazu gehört die mechanische Konstruktion des Roboters, einschließlich des Mechanismus zum Einwerfen der Spielsteine und die Überwachung des Spielfelds, sowie die Entwicklung der Software, die die Spielzüge und das Verhalten des Roboters steuert. Ein weiterer wichtiger Punkt ist die Sensorik: Der Roboter muss in der Lage sein, die aktuelle Spielsituation durch Farbsensoren zu erfassen, um zu wissen, welche Felder im Spielfeld bereits besetzt sind und wo er seinen nächsten Spielstein platzieren kann.
Zum Abschluss des Projekts wird ein kleines Turnier organisiert, bei dem die von verschiedenen Teams entwickelten Roboterlösungen gegeneinander antreten. Dies bietet die Möglichkeit, den Roboter unter realen Bedingungen zu testen und seine Fähigkeiten im direkten Vergleich mit den Lösungen der anderen Kommilitonen zu messen. Um zusätzliche Motivation zu schaffen, wird die beste Lösung am Ende prämiert, was den Wettbewerbsgedanken fördert und den Anreiz erhöht eine effektive Lösungen zu entwickeln.

\clearpage

\chapter*{Abstract} %*-Variante sorgt dafür, das Abstract nicht im Inhaltsverzeichnis auftaucht

English translation of the \glqq Kurzfassung\grqq.

\clearpage